\chapter{Conclusión}
\label{chap:conclusion}

Teniendo en cuenta el presente documento de análisis de seguridad del asistente virtual, se pueden observar varios puntos a tener en cuenta para mejorar el desarrollo realizado. En el presente documento se ha analizado desde un punto de vista de la seguridad del desarrollo: Arquitectura, Diseño, Implementación, Operaciones y Automatización y Tests.
En cada sección se han observado posibles riesgos y soluciones propuestas para subsanar y minimizar el riesgo del sistema. Entre los riesgos se destacan:

\begin{itemize}
    \item Correcto desarrollo de código. Se ha visto la necesidad de utilizar guías de estilo para desarrollo de código.
    \item Revisión de librerías de terceros: Hay librerías que no están actualizadas y se encuentran lejos de la última versión. Evitar este tipo de errores mejora la seguridad del proyecto.
    \item Falta de archivos logs. Es un sistema que se encuentra en producción y carece de archivos logs para comprobar si hay errores o no en producción.
    \item Análisis de código y pruebas.
    \item No hay entorno de preproducción. 
    
\end{itemize}

A nivel personal, el informe de seguridad ha ayudado para comprobar la seguridad del proyecto, así como utilizar los conceptos aprendidos en un proyecto real. La valoración es muy positiva y se volverá a aplicar en el resto de proyectos que tengo dentro de la empresa. Para concluir, el objetivo de la práctica ha sido completado: tanto analizar el código como conocer y aplicar esta metodología a un trabajo real (no es simulación).